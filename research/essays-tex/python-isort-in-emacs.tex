% Created 2020-02-04 Tue 17:20
% Intended LaTeX compiler: pdflatex
\documentclass[12pt, a4paper]{article}
\usepackage[utf8]{inputenc}
\usepackage[T1]{fontenc}
\usepackage{graphicx}
\usepackage{grffile}
\usepackage{longtable}
\usepackage{wrapfig}
\usepackage{rotating}
\usepackage[normalem]{ulem}
\usepackage{amsmath}
\usepackage{textcomp}
\usepackage{amssymb}
\usepackage{capt-of}
\usepackage{hyperref}
\usepackage[style=authoryear,natbib]{biblatex}
\setlength\bibitemsep{\baselineskip}
\addbibresource{/Users/guilhermesalome/Emacs/references.bib}
\usepackage[T1]{fontenc}
\usepackage{lmodern}
\usepackage{amsmath}
\usepackage{mathtools}
\usepackage{multirow}
\usepackage{booktabs}
\usepackage{bbm}
\usepackage{dsfont}
\usepackage[]{algorithm2e}
\usepackage[flushleft]{threeparttable}
\newcommand\numberthis{\addtocounter{equation}{1}\tag{\theequation}}
\newcommand{\E}[1]{\mathbb{E}{\left[#1\right]}}
\newcommand{\EQ}[1]{\mathbb{E}_t^{\mathbb{Q}}{\left[#1\right]}}
\newcommand{\EP}[1]{\mathbb{E}_t^{\mathbb{P}}{\left[#1\right]}}
\newcommand{\e}[1]{\text{e}^{#1}}
\newcommand{\abs}[1]{\left\vert{#1}\right\vert}
\newcommand{\dis}{\overset{d}{\sim}}
\newcommand{\Var}[1]{\mathrm{Var}\left(#1\right)}
\newcommand{\Corr}[1]{\mathrm{Corr}\left(#1\right)}
\newcommand{\Normal}[1]{\mathcal{N}\left(0, #1\right)}
\newcommand{\StdNormal}{\Normal{1}}
\newcommand{\stdnormal}{\mathcal{N}\left(0, 1\right)}
\newcommand{\Max}[1]{\text{max}\left\{#1\right\}}
\newcommand{\Set}[1]{\left\{#1\right\}}
\newcommand{\Norm}[1]{\left\Vert #1 \right\Vert}
\newcommand{\Prob}[1]{\mathbb{P}\left(#1\right)}
\newcommand{\ProbHat}[1]{\hat{\mathbb{P}}\left(#1\right)}
\renewcommand{\ln}[1]{\text{ln}\left(#1\right)}
\DeclareMathOperator*{\argmin}{\arg\!\min}
\DeclareMathOperator*{\argmax}{\arg\!\max}
\DeclarePairedDelimiter\ceil{\lceil}{\rceil}
\DeclarePairedDelimiter\floor{\lfloor}{\rfloor}
\newcommand{\Poisson}[1]{\text{Poisson}\left(#1\right)}
\newcommand{\Uniform}[1]{\text{Unif}#1}
\newcommand{\Cov}[1]{\mathrm{Cov}\left(#1\right)}
\usepackage[hang,small,bf]{caption}
\usepackage[margin=1in]{geometry}
\usepackage{mathtools}
\usepackage{xcolor}
\usepackage{resizegather}
\usepackage{multirow}
\definecolor{darkgreen}{rgb}{0.1, 0.6, 0.1}
\usepackage{float}
\usepackage{setspace}
\usepackage{listings}
\usepackage{import}
\input{/Users/guilhermesalome/Emacs/Settings/Latex/latex-code.tex}
\newtheorem{problem}{Problem}
\newtheorem{theorem}{Theorem}
\usepackage[flushleft]{threeparttable}
\date{}
\title{Python isort in Emacs}
\hypersetup{
 pdfauthor={Guilherme Salomé},
 pdftitle={Python isort in Emacs},
 pdfkeywords={},
 pdfsubject={},
 pdfcreator={Emacs 26.1 (Org mode 9.2.1)},
 pdflang={English}}
\begin{document}

\maketitle
The Python package \href{https://github.com/timothycrosley/isort}{isort} is awesome for organizing imports in \texttt{py} files in an uniform and predictable way.
Instead of worrying about what order and where to put your \texttt{import} statements, just code without any worries. When you are done, call \texttt{isort yourfile.py} on the terminal to have \texttt{isort} sort all of the imports.

I wrote a simple function that allows you to call \texttt{isort} from within Emacs, and have \texttt{isort} sort the imports in your current buffer.
The function can be added to your \texttt{init.el} file:
\lstset{language=elisp,label= ,caption= ,captionpos=b,numbers=none}
\begin{lstlisting}
(defun isort-buffer ()
  "Uses the Python program isort on current buffer."
  (interactive)
  (setq tmp-file (make-temp-file "isort-test"))
  (mark-whole-buffer)
  (write-region (point-min) (point-max) tmp-file)
  (shell-command (concat "isort " tmp-file))
  (erase-buffer)
  (insert-file-contents tmp-file))
\end{lstlisting}
To use it, simply call \texttt{M-x isort-buffer}.

I often use it in conjunction to \href{https://orgmode.org/manual/Working-with-source-code.html}{source code blocks} in \href{https://orgmode.org}{orgmode} when doing literate programming.
I use \texttt{org-edit-special} (usually assigned to \texttt{C-c '}) to open a Python source code block in a separate buffer, then call \texttt{M-x isort-buffer} to have \texttt{isort} organize all my imports in a neat way.
\end{document}